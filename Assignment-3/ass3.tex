\documentclass[11pt]{article}

    \usepackage[breakable]{tcolorbox}
    \usepackage{parskip} % Stop auto-indenting (to mimic markdown behaviour)
    
    \usepackage{iftex}
    \ifPDFTeX
    	\usepackage[T1]{fontenc}
    	\usepackage{mathpazo}
    \else
    	\usepackage{fontspec}
    \fi

    % Basic figure setup, for now with no caption control since it's done
    % automatically by Pandoc (which extracts ![](path) syntax from Markdown).
    \usepackage{graphicx}
    % Maintain compatibility with old templates. Remove in nbconvert 6.0
    \let\Oldincludegraphics\includegraphics
    % Ensure that by default, figures have no caption (until we provide a
    % proper Figure object with a Caption API and a way to capture that
    % in the conversion process - todo).
    \usepackage{caption}
    \DeclareCaptionFormat{nocaption}{}
    \captionsetup{format=nocaption,aboveskip=0pt,belowskip=0pt}

    \usepackage[Export]{adjustbox} % Used to constrain images to a maximum size
    \adjustboxset{max size={0.9\linewidth}{0.9\paperheight}}
    \usepackage{float}
    \floatplacement{figure}{H} % forces figures to be placed at the correct location
    \usepackage{xcolor} % Allow colors to be defined
    \usepackage{enumerate} % Needed for markdown enumerations to work
    \usepackage{geometry} % Used to adjust the document margins
    \usepackage{amsmath} % Equations
    \usepackage{amssymb} % Equations
    \usepackage{textcomp} % defines textquotesingle
    % Hack from http://tex.stackexchange.com/a/47451/13684:
    \AtBeginDocument{%
        \def\PYZsq{\textquotesingle}% Upright quotes in Pygmentized code
    }
    \usepackage{upquote} % Upright quotes for verbatim code
    \usepackage{eurosym} % defines \euro
    \usepackage[mathletters]{ucs} % Extended unicode (utf-8) support
    \usepackage{fancyvrb} % verbatim replacement that allows latex
    \usepackage{grffile} % extends the file name processing of package graphics 
                         % to support a larger range
    \makeatletter % fix for grffile with XeLaTeX
    \def\Gread@@xetex#1{%
      \IfFileExists{"\Gin@base".bb}%
      {\Gread@eps{\Gin@base.bb}}%
      {\Gread@@xetex@aux#1}%
    }
    \makeatother

    % The hyperref package gives us a pdf with properly built
    % internal navigation ('pdf bookmarks' for the table of contents,
    % internal cross-reference links, web links for URLs, etc.)
    \usepackage{hyperref}
    % The default LaTeX title has an obnoxious amount of whitespace. By default,
    % titling removes some of it. It also provides customization options.
    \usepackage{titling}
    \usepackage{longtable} % longtable support required by pandoc >1.10
    \usepackage{booktabs}  % table support for pandoc > 1.12.2
    \usepackage[inline]{enumitem} % IRkernel/repr support (it uses the enumerate* environment)
    \usepackage[normalem]{ulem} % ulem is needed to support strikethroughs (\sout)
                                % normalem makes italics be italics, not underlines
    \usepackage{mathrsfs}
    

    
    % Colors for the hyperref package
    \definecolor{urlcolor}{rgb}{0,.145,.698}
    \definecolor{linkcolor}{rgb}{.71,0.21,0.01}
    \definecolor{citecolor}{rgb}{.12,.54,.11}

    % ANSI colors
    \definecolor{ansi-black}{HTML}{3E424D}
    \definecolor{ansi-black-intense}{HTML}{282C36}
    \definecolor{ansi-red}{HTML}{E75C58}
    \definecolor{ansi-red-intense}{HTML}{B22B31}
    \definecolor{ansi-green}{HTML}{00A250}
    \definecolor{ansi-green-intense}{HTML}{007427}
    \definecolor{ansi-yellow}{HTML}{DDB62B}
    \definecolor{ansi-yellow-intense}{HTML}{B27D12}
    \definecolor{ansi-blue}{HTML}{208FFB}
    \definecolor{ansi-blue-intense}{HTML}{0065CA}
    \definecolor{ansi-magenta}{HTML}{D160C4}
    \definecolor{ansi-magenta-intense}{HTML}{A03196}
    \definecolor{ansi-cyan}{HTML}{60C6C8}
    \definecolor{ansi-cyan-intense}{HTML}{258F8F}
    \definecolor{ansi-white}{HTML}{C5C1B4}
    \definecolor{ansi-white-intense}{HTML}{A1A6B2}
    \definecolor{ansi-default-inverse-fg}{HTML}{FFFFFF}
    \definecolor{ansi-default-inverse-bg}{HTML}{000000}

    % commands and environments needed by pandoc snippets
    % extracted from the output of `pandoc -s`
    \providecommand{\tightlist}{%
      \setlength{\itemsep}{0pt}\setlength{\parskip}{0pt}}
    \DefineVerbatimEnvironment{Highlighting}{Verbatim}{commandchars=\\\{\}}
    % Add ',fontsize=\small' for more characters per line
    \newenvironment{Shaded}{}{}
    \newcommand{\KeywordTok}[1]{\textcolor[rgb]{0.00,0.44,0.13}{\textbf{{#1}}}}
    \newcommand{\DataTypeTok}[1]{\textcolor[rgb]{0.56,0.13,0.00}{{#1}}}
    \newcommand{\DecValTok}[1]{\textcolor[rgb]{0.25,0.63,0.44}{{#1}}}
    \newcommand{\BaseNTok}[1]{\textcolor[rgb]{0.25,0.63,0.44}{{#1}}}
    \newcommand{\FloatTok}[1]{\textcolor[rgb]{0.25,0.63,0.44}{{#1}}}
    \newcommand{\CharTok}[1]{\textcolor[rgb]{0.25,0.44,0.63}{{#1}}}
    \newcommand{\StringTok}[1]{\textcolor[rgb]{0.25,0.44,0.63}{{#1}}}
    \newcommand{\CommentTok}[1]{\textcolor[rgb]{0.38,0.63,0.69}{\textit{{#1}}}}
    \newcommand{\OtherTok}[1]{\textcolor[rgb]{0.00,0.44,0.13}{{#1}}}
    \newcommand{\AlertTok}[1]{\textcolor[rgb]{1.00,0.00,0.00}{\textbf{{#1}}}}
    \newcommand{\FunctionTok}[1]{\textcolor[rgb]{0.02,0.16,0.49}{{#1}}}
    \newcommand{\RegionMarkerTok}[1]{{#1}}
    \newcommand{\ErrorTok}[1]{\textcolor[rgb]{1.00,0.00,0.00}{\textbf{{#1}}}}
    \newcommand{\NormalTok}[1]{{#1}}
    
    % Additional commands for more recent versions of Pandoc
    \newcommand{\ConstantTok}[1]{\textcolor[rgb]{0.53,0.00,0.00}{{#1}}}
    \newcommand{\SpecialCharTok}[1]{\textcolor[rgb]{0.25,0.44,0.63}{{#1}}}
    \newcommand{\VerbatimStringTok}[1]{\textcolor[rgb]{0.25,0.44,0.63}{{#1}}}
    \newcommand{\SpecialStringTok}[1]{\textcolor[rgb]{0.73,0.40,0.53}{{#1}}}
    \newcommand{\ImportTok}[1]{{#1}}
    \newcommand{\DocumentationTok}[1]{\textcolor[rgb]{0.73,0.13,0.13}{\textit{{#1}}}}
    \newcommand{\AnnotationTok}[1]{\textcolor[rgb]{0.38,0.63,0.69}{\textbf{\textit{{#1}}}}}
    \newcommand{\CommentVarTok}[1]{\textcolor[rgb]{0.38,0.63,0.69}{\textbf{\textit{{#1}}}}}
    \newcommand{\VariableTok}[1]{\textcolor[rgb]{0.10,0.09,0.49}{{#1}}}
    \newcommand{\ControlFlowTok}[1]{\textcolor[rgb]{0.00,0.44,0.13}{\textbf{{#1}}}}
    \newcommand{\OperatorTok}[1]{\textcolor[rgb]{0.40,0.40,0.40}{{#1}}}
    \newcommand{\BuiltInTok}[1]{{#1}}
    \newcommand{\ExtensionTok}[1]{{#1}}
    \newcommand{\PreprocessorTok}[1]{\textcolor[rgb]{0.74,0.48,0.00}{{#1}}}
    \newcommand{\AttributeTok}[1]{\textcolor[rgb]{0.49,0.56,0.16}{{#1}}}
    \newcommand{\InformationTok}[1]{\textcolor[rgb]{0.38,0.63,0.69}{\textbf{\textit{{#1}}}}}
    \newcommand{\WarningTok}[1]{\textcolor[rgb]{0.38,0.63,0.69}{\textbf{\textit{{#1}}}}}
    
    
    % Define a nice break command that doesn't care if a line doesn't already
    % exist.
    \def\br{\hspace*{\fill} \\* }
    % Math Jax compatibility definitions
    \def\gt{>}
    \def\lt{<}
    \let\Oldtex\TeX
    \let\Oldlatex\LaTeX
    \renewcommand{\TeX}{\textrm{\Oldtex}}
    \renewcommand{\LaTeX}{\textrm{\Oldlatex}}
    % Document parameters
    % Document title
    \title{Untitled}
    
    
    
    
    
% Pygments definitions
\makeatletter
\def\PY@reset{\let\PY@it=\relax \let\PY@bf=\relax%
    \let\PY@ul=\relax \let\PY@tc=\relax%
    \let\PY@bc=\relax \let\PY@ff=\relax}
\def\PY@tok#1{\csname PY@tok@#1\endcsname}
\def\PY@toks#1+{\ifx\relax#1\empty\else%
    \PY@tok{#1}\expandafter\PY@toks\fi}
\def\PY@do#1{\PY@bc{\PY@tc{\PY@ul{%
    \PY@it{\PY@bf{\PY@ff{#1}}}}}}}
\def\PY#1#2{\PY@reset\PY@toks#1+\relax+\PY@do{#2}}

\expandafter\def\csname PY@tok@w\endcsname{\def\PY@tc##1{\textcolor[rgb]{0.73,0.73,0.73}{##1}}}
\expandafter\def\csname PY@tok@c\endcsname{\let\PY@it=\textit\def\PY@tc##1{\textcolor[rgb]{0.25,0.50,0.50}{##1}}}
\expandafter\def\csname PY@tok@cp\endcsname{\def\PY@tc##1{\textcolor[rgb]{0.74,0.48,0.00}{##1}}}
\expandafter\def\csname PY@tok@k\endcsname{\let\PY@bf=\textbf\def\PY@tc##1{\textcolor[rgb]{0.00,0.50,0.00}{##1}}}
\expandafter\def\csname PY@tok@kp\endcsname{\def\PY@tc##1{\textcolor[rgb]{0.00,0.50,0.00}{##1}}}
\expandafter\def\csname PY@tok@kt\endcsname{\def\PY@tc##1{\textcolor[rgb]{0.69,0.00,0.25}{##1}}}
\expandafter\def\csname PY@tok@o\endcsname{\def\PY@tc##1{\textcolor[rgb]{0.40,0.40,0.40}{##1}}}
\expandafter\def\csname PY@tok@ow\endcsname{\let\PY@bf=\textbf\def\PY@tc##1{\textcolor[rgb]{0.67,0.13,1.00}{##1}}}
\expandafter\def\csname PY@tok@nb\endcsname{\def\PY@tc##1{\textcolor[rgb]{0.00,0.50,0.00}{##1}}}
\expandafter\def\csname PY@tok@nf\endcsname{\def\PY@tc##1{\textcolor[rgb]{0.00,0.00,1.00}{##1}}}
\expandafter\def\csname PY@tok@nc\endcsname{\let\PY@bf=\textbf\def\PY@tc##1{\textcolor[rgb]{0.00,0.00,1.00}{##1}}}
\expandafter\def\csname PY@tok@nn\endcsname{\let\PY@bf=\textbf\def\PY@tc##1{\textcolor[rgb]{0.00,0.00,1.00}{##1}}}
\expandafter\def\csname PY@tok@ne\endcsname{\let\PY@bf=\textbf\def\PY@tc##1{\textcolor[rgb]{0.82,0.25,0.23}{##1}}}
\expandafter\def\csname PY@tok@nv\endcsname{\def\PY@tc##1{\textcolor[rgb]{0.10,0.09,0.49}{##1}}}
\expandafter\def\csname PY@tok@no\endcsname{\def\PY@tc##1{\textcolor[rgb]{0.53,0.00,0.00}{##1}}}
\expandafter\def\csname PY@tok@nl\endcsname{\def\PY@tc##1{\textcolor[rgb]{0.63,0.63,0.00}{##1}}}
\expandafter\def\csname PY@tok@ni\endcsname{\let\PY@bf=\textbf\def\PY@tc##1{\textcolor[rgb]{0.60,0.60,0.60}{##1}}}
\expandafter\def\csname PY@tok@na\endcsname{\def\PY@tc##1{\textcolor[rgb]{0.49,0.56,0.16}{##1}}}
\expandafter\def\csname PY@tok@nt\endcsname{\let\PY@bf=\textbf\def\PY@tc##1{\textcolor[rgb]{0.00,0.50,0.00}{##1}}}
\expandafter\def\csname PY@tok@nd\endcsname{\def\PY@tc##1{\textcolor[rgb]{0.67,0.13,1.00}{##1}}}
\expandafter\def\csname PY@tok@s\endcsname{\def\PY@tc##1{\textcolor[rgb]{0.73,0.13,0.13}{##1}}}
\expandafter\def\csname PY@tok@sd\endcsname{\let\PY@it=\textit\def\PY@tc##1{\textcolor[rgb]{0.73,0.13,0.13}{##1}}}
\expandafter\def\csname PY@tok@si\endcsname{\let\PY@bf=\textbf\def\PY@tc##1{\textcolor[rgb]{0.73,0.40,0.53}{##1}}}
\expandafter\def\csname PY@tok@se\endcsname{\let\PY@bf=\textbf\def\PY@tc##1{\textcolor[rgb]{0.73,0.40,0.13}{##1}}}
\expandafter\def\csname PY@tok@sr\endcsname{\def\PY@tc##1{\textcolor[rgb]{0.73,0.40,0.53}{##1}}}
\expandafter\def\csname PY@tok@ss\endcsname{\def\PY@tc##1{\textcolor[rgb]{0.10,0.09,0.49}{##1}}}
\expandafter\def\csname PY@tok@sx\endcsname{\def\PY@tc##1{\textcolor[rgb]{0.00,0.50,0.00}{##1}}}
\expandafter\def\csname PY@tok@m\endcsname{\def\PY@tc##1{\textcolor[rgb]{0.40,0.40,0.40}{##1}}}
\expandafter\def\csname PY@tok@gh\endcsname{\let\PY@bf=\textbf\def\PY@tc##1{\textcolor[rgb]{0.00,0.00,0.50}{##1}}}
\expandafter\def\csname PY@tok@gu\endcsname{\let\PY@bf=\textbf\def\PY@tc##1{\textcolor[rgb]{0.50,0.00,0.50}{##1}}}
\expandafter\def\csname PY@tok@gd\endcsname{\def\PY@tc##1{\textcolor[rgb]{0.63,0.00,0.00}{##1}}}
\expandafter\def\csname PY@tok@gi\endcsname{\def\PY@tc##1{\textcolor[rgb]{0.00,0.63,0.00}{##1}}}
\expandafter\def\csname PY@tok@gr\endcsname{\def\PY@tc##1{\textcolor[rgb]{1.00,0.00,0.00}{##1}}}
\expandafter\def\csname PY@tok@ge\endcsname{\let\PY@it=\textit}
\expandafter\def\csname PY@tok@gs\endcsname{\let\PY@bf=\textbf}
\expandafter\def\csname PY@tok@gp\endcsname{\let\PY@bf=\textbf\def\PY@tc##1{\textcolor[rgb]{0.00,0.00,0.50}{##1}}}
\expandafter\def\csname PY@tok@go\endcsname{\def\PY@tc##1{\textcolor[rgb]{0.53,0.53,0.53}{##1}}}
\expandafter\def\csname PY@tok@gt\endcsname{\def\PY@tc##1{\textcolor[rgb]{0.00,0.27,0.87}{##1}}}
\expandafter\def\csname PY@tok@err\endcsname{\def\PY@bc##1{\setlength{\fboxsep}{0pt}\fcolorbox[rgb]{1.00,0.00,0.00}{1,1,1}{\strut ##1}}}
\expandafter\def\csname PY@tok@kc\endcsname{\let\PY@bf=\textbf\def\PY@tc##1{\textcolor[rgb]{0.00,0.50,0.00}{##1}}}
\expandafter\def\csname PY@tok@kd\endcsname{\let\PY@bf=\textbf\def\PY@tc##1{\textcolor[rgb]{0.00,0.50,0.00}{##1}}}
\expandafter\def\csname PY@tok@kn\endcsname{\let\PY@bf=\textbf\def\PY@tc##1{\textcolor[rgb]{0.00,0.50,0.00}{##1}}}
\expandafter\def\csname PY@tok@kr\endcsname{\let\PY@bf=\textbf\def\PY@tc##1{\textcolor[rgb]{0.00,0.50,0.00}{##1}}}
\expandafter\def\csname PY@tok@bp\endcsname{\def\PY@tc##1{\textcolor[rgb]{0.00,0.50,0.00}{##1}}}
\expandafter\def\csname PY@tok@fm\endcsname{\def\PY@tc##1{\textcolor[rgb]{0.00,0.00,1.00}{##1}}}
\expandafter\def\csname PY@tok@vc\endcsname{\def\PY@tc##1{\textcolor[rgb]{0.10,0.09,0.49}{##1}}}
\expandafter\def\csname PY@tok@vg\endcsname{\def\PY@tc##1{\textcolor[rgb]{0.10,0.09,0.49}{##1}}}
\expandafter\def\csname PY@tok@vi\endcsname{\def\PY@tc##1{\textcolor[rgb]{0.10,0.09,0.49}{##1}}}
\expandafter\def\csname PY@tok@vm\endcsname{\def\PY@tc##1{\textcolor[rgb]{0.10,0.09,0.49}{##1}}}
\expandafter\def\csname PY@tok@sa\endcsname{\def\PY@tc##1{\textcolor[rgb]{0.73,0.13,0.13}{##1}}}
\expandafter\def\csname PY@tok@sb\endcsname{\def\PY@tc##1{\textcolor[rgb]{0.73,0.13,0.13}{##1}}}
\expandafter\def\csname PY@tok@sc\endcsname{\def\PY@tc##1{\textcolor[rgb]{0.73,0.13,0.13}{##1}}}
\expandafter\def\csname PY@tok@dl\endcsname{\def\PY@tc##1{\textcolor[rgb]{0.73,0.13,0.13}{##1}}}
\expandafter\def\csname PY@tok@s2\endcsname{\def\PY@tc##1{\textcolor[rgb]{0.73,0.13,0.13}{##1}}}
\expandafter\def\csname PY@tok@sh\endcsname{\def\PY@tc##1{\textcolor[rgb]{0.73,0.13,0.13}{##1}}}
\expandafter\def\csname PY@tok@s1\endcsname{\def\PY@tc##1{\textcolor[rgb]{0.73,0.13,0.13}{##1}}}
\expandafter\def\csname PY@tok@mb\endcsname{\def\PY@tc##1{\textcolor[rgb]{0.40,0.40,0.40}{##1}}}
\expandafter\def\csname PY@tok@mf\endcsname{\def\PY@tc##1{\textcolor[rgb]{0.40,0.40,0.40}{##1}}}
\expandafter\def\csname PY@tok@mh\endcsname{\def\PY@tc##1{\textcolor[rgb]{0.40,0.40,0.40}{##1}}}
\expandafter\def\csname PY@tok@mi\endcsname{\def\PY@tc##1{\textcolor[rgb]{0.40,0.40,0.40}{##1}}}
\expandafter\def\csname PY@tok@il\endcsname{\def\PY@tc##1{\textcolor[rgb]{0.40,0.40,0.40}{##1}}}
\expandafter\def\csname PY@tok@mo\endcsname{\def\PY@tc##1{\textcolor[rgb]{0.40,0.40,0.40}{##1}}}
\expandafter\def\csname PY@tok@ch\endcsname{\let\PY@it=\textit\def\PY@tc##1{\textcolor[rgb]{0.25,0.50,0.50}{##1}}}
\expandafter\def\csname PY@tok@cm\endcsname{\let\PY@it=\textit\def\PY@tc##1{\textcolor[rgb]{0.25,0.50,0.50}{##1}}}
\expandafter\def\csname PY@tok@cpf\endcsname{\let\PY@it=\textit\def\PY@tc##1{\textcolor[rgb]{0.25,0.50,0.50}{##1}}}
\expandafter\def\csname PY@tok@c1\endcsname{\let\PY@it=\textit\def\PY@tc##1{\textcolor[rgb]{0.25,0.50,0.50}{##1}}}
\expandafter\def\csname PY@tok@cs\endcsname{\let\PY@it=\textit\def\PY@tc##1{\textcolor[rgb]{0.25,0.50,0.50}{##1}}}

\def\PYZbs{\char`\\}
\def\PYZus{\char`\_}
\def\PYZob{\char`\{}
\def\PYZcb{\char`\}}
\def\PYZca{\char`\^}
\def\PYZam{\char`\&}
\def\PYZlt{\char`\<}
\def\PYZgt{\char`\>}
\def\PYZsh{\char`\#}
\def\PYZpc{\char`\%}
\def\PYZdl{\char`\$}
\def\PYZhy{\char`\-}
\def\PYZsq{\char`\'}
\def\PYZdq{\char`\"}
\def\PYZti{\char`\~}
% for compatibility with earlier versions
\def\PYZat{@}
\def\PYZlb{[}
\def\PYZrb{]}
\makeatother


    % For linebreaks inside Verbatim environment from package fancyvrb. 
    \makeatletter
        \newbox\Wrappedcontinuationbox 
        \newbox\Wrappedvisiblespacebox 
        \newcommand*\Wrappedvisiblespace {\textcolor{red}{\textvisiblespace}} 
        \newcommand*\Wrappedcontinuationsymbol {\textcolor{red}{\llap{\tiny$\m@th\hookrightarrow$}}} 
        \newcommand*\Wrappedcontinuationindent {3ex } 
        \newcommand*\Wrappedafterbreak {\kern\Wrappedcontinuationindent\copy\Wrappedcontinuationbox} 
        % Take advantage of the already applied Pygments mark-up to insert 
        % potential linebreaks for TeX processing. 
        %        {, <, #, %, $, ' and ": go to next line. 
        %        _, }, ^, &, >, - and ~: stay at end of broken line. 
        % Use of \textquotesingle for straight quote. 
        \newcommand*\Wrappedbreaksatspecials {% 
            \def\PYGZus{\discretionary{\char`\_}{\Wrappedafterbreak}{\char`\_}}% 
            \def\PYGZob{\discretionary{}{\Wrappedafterbreak\char`\{}{\char`\{}}% 
            \def\PYGZcb{\discretionary{\char`\}}{\Wrappedafterbreak}{\char`\}}}% 
            \def\PYGZca{\discretionary{\char`\^}{\Wrappedafterbreak}{\char`\^}}% 
            \def\PYGZam{\discretionary{\char`\&}{\Wrappedafterbreak}{\char`\&}}% 
            \def\PYGZlt{\discretionary{}{\Wrappedafterbreak\char`\<}{\char`\<}}% 
            \def\PYGZgt{\discretionary{\char`\>}{\Wrappedafterbreak}{\char`\>}}% 
            \def\PYGZsh{\discretionary{}{\Wrappedafterbreak\char`\#}{\char`\#}}% 
            \def\PYGZpc{\discretionary{}{\Wrappedafterbreak\char`\%}{\char`\%}}% 
            \def\PYGZdl{\discretionary{}{\Wrappedafterbreak\char`\$}{\char`\$}}% 
            \def\PYGZhy{\discretionary{\char`\-}{\Wrappedafterbreak}{\char`\-}}% 
            \def\PYGZsq{\discretionary{}{\Wrappedafterbreak\textquotesingle}{\textquotesingle}}% 
            \def\PYGZdq{\discretionary{}{\Wrappedafterbreak\char`\"}{\char`\"}}% 
            \def\PYGZti{\discretionary{\char`\~}{\Wrappedafterbreak}{\char`\~}}% 
        } 
        % Some characters . , ; ? ! / are not pygmentized. 
        % This macro makes them "active" and they will insert potential linebreaks 
        \newcommand*\Wrappedbreaksatpunct {% 
            \lccode`\~`\.\lowercase{\def~}{\discretionary{\hbox{\char`\.}}{\Wrappedafterbreak}{\hbox{\char`\.}}}% 
            \lccode`\~`\,\lowercase{\def~}{\discretionary{\hbox{\char`\,}}{\Wrappedafterbreak}{\hbox{\char`\,}}}% 
            \lccode`\~`\;\lowercase{\def~}{\discretionary{\hbox{\char`\;}}{\Wrappedafterbreak}{\hbox{\char`\;}}}% 
            \lccode`\~`\:\lowercase{\def~}{\discretionary{\hbox{\char`\:}}{\Wrappedafterbreak}{\hbox{\char`\:}}}% 
            \lccode`\~`\?\lowercase{\def~}{\discretionary{\hbox{\char`\?}}{\Wrappedafterbreak}{\hbox{\char`\?}}}% 
            \lccode`\~`\!\lowercase{\def~}{\discretionary{\hbox{\char`\!}}{\Wrappedafterbreak}{\hbox{\char`\!}}}% 
            \lccode`\~`\/\lowercase{\def~}{\discretionary{\hbox{\char`\/}}{\Wrappedafterbreak}{\hbox{\char`\/}}}% 
            \catcode`\.\active
            \catcode`\,\active 
            \catcode`\;\active
            \catcode`\:\active
            \catcode`\?\active
            \catcode`\!\active
            \catcode`\/\active 
            \lccode`\~`\~ 	
        }
    \makeatother

    \let\OriginalVerbatim=\Verbatim
    \makeatletter
    \renewcommand{\Verbatim}[1][1]{%
        %\parskip\z@skip
        \sbox\Wrappedcontinuationbox {\Wrappedcontinuationsymbol}%
        \sbox\Wrappedvisiblespacebox {\FV@SetupFont\Wrappedvisiblespace}%
        \def\FancyVerbFormatLine ##1{\hsize\linewidth
            \vtop{\raggedright\hyphenpenalty\z@\exhyphenpenalty\z@
                \doublehyphendemerits\z@\finalhyphendemerits\z@
                \strut ##1\strut}%
        }%
        % If the linebreak is at a space, the latter will be displayed as visible
        % space at end of first line, and a continuation symbol starts next line.
        % Stretch/shrink are however usually zero for typewriter font.
        \def\FV@Space {%
            \nobreak\hskip\z@ plus\fontdimen3\font minus\fontdimen4\font
            \discretionary{\copy\Wrappedvisiblespacebox}{\Wrappedafterbreak}
            {\kern\fontdimen2\font}%
        }%
        
        % Allow breaks at special characters using \PYG... macros.
        \Wrappedbreaksatspecials
        % Breaks at punctuation characters . , ; ? ! and / need catcode=\active 	
        \OriginalVerbatim[#1,codes*=\Wrappedbreaksatpunct]%
    }
    \makeatother

    % Exact colors from NB
    \definecolor{incolor}{HTML}{303F9F}
    \definecolor{outcolor}{HTML}{D84315}
    \definecolor{cellborder}{HTML}{CFCFCF}
    \definecolor{cellbackground}{HTML}{F7F7F7}
    
    % prompt
    \makeatletter
    \newcommand{\boxspacing}{\kern\kvtcb@left@rule\kern\kvtcb@boxsep}
    \makeatother
    \newcommand{\prompt}[4]{
        \ttfamily\llap{{\color{#2}[#3]:\hspace{3pt}#4}}\vspace{-\baselineskip}
    }
    

    
    % Prevent overflowing lines due to hard-to-break entities
    \sloppy 
    % Setup hyperref package
    \hypersetup{
      breaklinks=true,  % so long urls are correctly broken across lines
      colorlinks=true,
      urlcolor=urlcolor,
      linkcolor=linkcolor,
      citecolor=citecolor,
      }
    % Slightly bigger margins than the latex defaults
    
    \geometry{verbose,tmargin=1in,bmargin=1in,lmargin=1in,rmargin=1in}
    
    

\begin{document}
    
    \maketitle
    
    

    
    \hypertarget{aim-of-the-experiment}{%
\section{Aim of the experiment}\label{aim-of-the-experiment}}

In this assignment we aim to observe the error made by the least-mean
square fit function from the scipy library and observe the relation
between error and noise.

    \hypertarget{procedure}{%
\section{Procedure}\label{procedure}}

    \hypertarget{part-0}{%
\subsection{Part 0}\label{part-0}}

Importing required libraries.

    \begin{tcolorbox}[breakable, size=fbox, boxrule=1pt, pad at break*=1mm,colback=cellbackground, colframe=cellborder]
\prompt{In}{incolor}{1}{\boxspacing}
\begin{Verbatim}[commandchars=\\\{\}]
\PY{k+kn}{from} \PY{n+nn}{pylab} \PY{k}{import} \PY{o}{*}
\PY{k+kn}{import} \PY{n+nn}{matplotlib}\PY{n+nn}{.}\PY{n+nn}{pyplot} \PY{k}{as} \PY{n+nn}{plt}
\PY{k+kn}{import} \PY{n+nn}{scipy}\PY{n+nn}{.}\PY{n+nn}{special} \PY{k}{as} \PY{n+nn}{sp}
\PY{k+kn}{import} \PY{n+nn}{scipy}\PY{n+nn}{.}\PY{n+nn}{linalg} \PY{k}{as} \PY{n+nn}{linalg}
\PY{k+kn}{import} \PY{n+nn}{time}
\end{Verbatim}
\end{tcolorbox}

    \hypertarget{part-1-3}{%
\subsection{Part 1-3}\label{part-1-3}}

The given code is excecuted to generate noisy data of linear combination
of `bessel' function. \emph{f(t) = 1.05J(t)-0.105t}

    \begin{tcolorbox}[breakable, size=fbox, boxrule=1pt, pad at break*=1mm,colback=cellbackground, colframe=cellborder]
\prompt{In}{incolor}{2}{\boxspacing}
\begin{Verbatim}[commandchars=\\\{\}]
\PY{n}{N}\PY{o}{=}\PY{l+m+mi}{101} \PY{c+c1}{\PYZsh{} no of data points}
\PY{n}{k}\PY{o}{=}\PY{l+m+mi}{9} \PY{c+c1}{\PYZsh{} no of sets of data with varying noise}
\PY{c+c1}{\PYZsh{} generate the data points and add noise}

\PY{n}{t}\PY{o}{=}\PY{n}{linspace}\PY{p}{(}\PY{l+m+mi}{0}\PY{p}{,}\PY{l+m+mi}{10}\PY{p}{,}\PY{n}{N}\PY{p}{)} \PY{c+c1}{\PYZsh{} t vector}
\PY{n}{y}\PY{o}{=}\PY{l+m+mf}{1.05}\PY{o}{*}\PY{n}{sp}\PY{o}{.}\PY{n}{jv}\PY{p}{(}\PY{l+m+mi}{2}\PY{p}{,}\PY{n}{t}\PY{p}{)}\PY{o}{\PYZhy{}}\PY{l+m+mf}{0.105}\PY{o}{*}\PY{n}{t} \PY{c+c1}{\PYZsh{} f(t) vector}
\PY{n}{Y}\PY{o}{=}\PY{n}{meshgrid}\PY{p}{(}\PY{n}{y}\PY{p}{,}\PY{n}{ones}\PY{p}{(}\PY{n}{k}\PY{p}{)}\PY{p}{,}\PY{n}{indexing}\PY{o}{=}\PY{l+s+s1}{\PYZsq{}}\PY{l+s+s1}{ij}\PY{l+s+s1}{\PYZsq{}}\PY{p}{)}\PY{p}{[}\PY{l+m+mi}{0}\PY{p}{]} \PY{c+c1}{\PYZsh{} make k copies}
\PY{n}{scl}\PY{o}{=}\PY{n}{logspace}\PY{p}{(}\PY{o}{\PYZhy{}}\PY{l+m+mi}{1}\PY{p}{,}\PY{o}{\PYZhy{}}\PY{l+m+mi}{3}\PY{p}{,}\PY{n}{k}\PY{p}{)} \PY{c+c1}{\PYZsh{} noise stdev}
\PY{n}{n}\PY{o}{=}\PY{n}{dot}\PY{p}{(}\PY{n}{randn}\PY{p}{(}\PY{n}{N}\PY{p}{,}\PY{n}{k}\PY{p}{)}\PY{p}{,}\PY{n}{diag}\PY{p}{(}\PY{n}{scl}\PY{p}{)}\PY{p}{)} \PY{c+c1}{\PYZsh{} generate k vectors}
\PY{n}{yy}\PY{o}{=}\PY{n}{Y}\PY{o}{+}\PY{n}{n} \PY{c+c1}{\PYZsh{} add noise to signal}
\end{Verbatim}
\end{tcolorbox}

    \hypertarget{part-4}{%
\subsection{Part 4}\label{part-4}}

A function g(t) is written which takes two arguments A,B, and returns a
answer of the linear combination of `bessel' function. Then the noisy
data is plotted along with the true value of the function.

    \begin{tcolorbox}[breakable, size=fbox, boxrule=1pt, pad at break*=1mm,colback=cellbackground, colframe=cellborder]
\prompt{In}{incolor}{3}{\boxspacing}
\begin{Verbatim}[commandchars=\\\{\}]
\PY{k}{def} \PY{n+nf}{g}\PY{p}{(}\PY{n}{t}\PY{p}{,}\PY{n}{A}\PY{o}{=}\PY{l+m+mf}{1.05}\PY{p}{,}\PY{n}{B}\PY{o}{=}\PY{o}{\PYZhy{}}\PY{l+m+mf}{0.105}\PY{p}{)}\PY{p}{:}
    \PY{n}{y}\PY{o}{=}\PY{n}{A}\PY{o}{*}\PY{n}{sp}\PY{o}{.}\PY{n}{jv}\PY{p}{(}\PY{l+m+mi}{2}\PY{p}{,}\PY{n}{t}\PY{p}{)}\PY{o}{+}\PY{n}{B}\PY{o}{*}\PY{n}{t}
    \PY{k}{return} \PY{n}{y}
\end{Verbatim}
\end{tcolorbox}

    \begin{tcolorbox}[breakable, size=fbox, boxrule=1pt, pad at break*=1mm,colback=cellbackground, colframe=cellborder]
\prompt{In}{incolor}{4}{\boxspacing}
\begin{Verbatim}[commandchars=\\\{\}]
\PY{n}{arr} \PY{o}{=} \PY{p}{[}\PY{p}{]}
\PY{k}{for} \PY{n}{i} \PY{o+ow}{in} \PY{n+nb}{range}\PY{p}{(}\PY{l+m+mi}{9}\PY{p}{)}\PY{p}{:}
    \PY{n}{arr}\PY{o}{.}\PY{n}{append}\PY{p}{(}\PY{l+s+s1}{\PYZsq{}}\PY{l+s+se}{\PYZbs{}u03C3}\PY{l+s+s1}{\PYZsq{}}\PY{o}{+}\PY{n+nb}{str}\PY{p}{(}\PY{n}{i}\PY{o}{+}\PY{l+m+mi}{1}\PY{p}{)}\PY{o}{+}\PY{l+s+s1}{\PYZsq{}}\PY{l+s+s1}{ = }\PY{l+s+s1}{\PYZsq{}}\PY{o}{+}\PY{n+nb}{str}\PY{p}{(}\PY{n+nb}{round}\PY{p}{(}\PY{n}{scl}\PY{p}{[}\PY{n}{i}\PY{p}{]}\PY{p}{,}\PY{l+m+mi}{3}\PY{p}{)}\PY{p}{)}\PY{p}{)}
\PY{n}{arr}\PY{o}{.}\PY{n}{append}\PY{p}{(}\PY{l+s+s1}{\PYZsq{}}\PY{l+s+s1}{True Value}\PY{l+s+s1}{\PYZsq{}}\PY{p}{)}

\PY{c+c1}{\PYZsh{}true value of the function without noise as calculated by the function g }
\PY{n}{gt} \PY{o}{=} \PY{n}{np}\PY{o}{.}\PY{n}{array}\PY{p}{(}\PY{n}{g}\PY{p}{(}\PY{n}{t}\PY{p}{)}\PY{p}{)}\PY{p}{[}\PY{p}{:}\PY{p}{,} \PY{k+kc}{None}\PY{p}{]}
\end{Verbatim}
\end{tcolorbox}

    \begin{tcolorbox}[breakable, size=fbox, boxrule=1pt, pad at break*=1mm,colback=cellbackground, colframe=cellborder]
\prompt{In}{incolor}{5}{\boxspacing}
\begin{Verbatim}[commandchars=\\\{\}]
\PY{c+c1}{\PYZsh{}gt contains true value of the function without noise}
\PY{n}{y} \PY{o}{=} \PY{n}{np}\PY{o}{.}\PY{n}{concatenate}\PY{p}{(}\PY{p}{(}\PY{n}{yy}\PY{p}{,}\PY{n}{gt}\PY{p}{)}\PY{p}{,}\PY{n}{axis}\PY{o}{=}\PY{l+m+mi}{1}\PY{p}{)}
\PY{c+c1}{\PYZsh{}plotting the noisy data and true value of the function}
\PY{n}{lines} \PY{o}{=} \PY{n}{plot}\PY{p}{(}\PY{n}{t}\PY{p}{,} \PY{n}{y}\PY{p}{)}
\PY{n}{legend}\PY{p}{(}\PY{n}{lines}\PY{p}{,} \PY{n}{arr}\PY{p}{)}
\PY{n}{xlabel}\PY{p}{(}\PY{l+s+sa}{r}\PY{l+s+s1}{\PYZsq{}}\PY{l+s+s1}{\PYZdl{}t\PYZdl{}}\PY{l+s+s1}{\PYZsq{}}\PY{p}{,}\PY{n}{size}\PY{o}{=}\PY{l+m+mi}{20}\PY{p}{)}
\PY{n}{ylabel}\PY{p}{(}\PY{l+s+sa}{r}\PY{l+s+s1}{\PYZsq{}}\PY{l+s+s1}{\PYZdl{}f(t)+n\PYZdl{}}\PY{l+s+s1}{\PYZsq{}}\PY{p}{,}\PY{n}{size}\PY{o}{=}\PY{l+m+mi}{20}\PY{p}{)}
\PY{n}{title}\PY{p}{(}\PY{l+s+sa}{r}\PY{l+s+s1}{\PYZsq{}}\PY{l+s+s1}{Figure 0}\PY{l+s+s1}{\PYZsq{}}\PY{p}{)}
\PY{n}{grid}\PY{p}{(}\PY{k+kc}{True}\PY{p}{)}
\PY{c+c1}{\PYZsh{}savetxt(\PYZdq{}fitting.dat\PYZdq{},c\PYZus{}[t,yy]) \PYZsh{} write out matrix to file}
\PY{n}{show}\PY{p}{(}\PY{p}{)}
\end{Verbatim}
\end{tcolorbox}

    \begin{center}
    \adjustimage{max size={0.9\linewidth}{0.9\paperheight}}{output_9_0.png}
    \end{center}
    { \hspace*{\fill} \\}
    
    \hypertarget{part-5}{%
\subsection{Part 5}\label{part-5}}

Plotting errorbar for a noise containing sigma value as 0.1. Errorbar
gives us the possible range of error in the plot.

    \begin{tcolorbox}[breakable, size=fbox, boxrule=1pt, pad at break*=1mm,colback=cellbackground, colframe=cellborder]
\prompt{In}{incolor}{6}{\boxspacing}
\begin{Verbatim}[commandchars=\\\{\}]
\PY{n}{data} \PY{o}{=} \PY{n}{yy}
\PY{c+c1}{\PYZsh{}plotting error bar}
\PY{n}{errorbar}\PY{p}{(}\PY{n}{t}\PY{p}{[}\PY{p}{:}\PY{p}{:}\PY{l+m+mi}{5}\PY{p}{]}\PY{p}{,}\PY{n}{yy}\PY{p}{[}\PY{p}{:}\PY{p}{,}\PY{l+m+mi}{0}\PY{p}{]}\PY{p}{[}\PY{p}{:}\PY{p}{:}\PY{l+m+mi}{5}\PY{p}{]}\PY{p}{,}\PY{l+m+mf}{0.1}\PY{p}{,}\PY{n}{fmt}\PY{o}{=}\PY{l+s+s1}{\PYZsq{}}\PY{l+s+s1}{ro}\PY{l+s+s1}{\PYZsq{}}\PY{p}{,}\PY{n}{label}\PY{o}{=}\PY{l+s+s1}{\PYZsq{}}\PY{l+s+s1}{err bar}\PY{l+s+s1}{\PYZsq{}}\PY{p}{)}
\PY{c+c1}{\PYZsh{}plotting the true value}
\PY{n}{plot}\PY{p}{(}\PY{n}{t}\PY{p}{,}\PY{n}{gt}\PY{p}{,}\PY{n}{label}\PY{o}{=}\PY{l+s+s1}{\PYZsq{}}\PY{l+s+s1}{f(t)}\PY{l+s+s1}{\PYZsq{}}\PY{p}{)}
\PY{n}{xlabel}\PY{p}{(}\PY{l+s+sa}{r}\PY{l+s+s1}{\PYZsq{}}\PY{l+s+s1}{\PYZdl{}t\PYZdl{}}\PY{l+s+s1}{\PYZsq{}}\PY{p}{,}\PY{n}{size}\PY{o}{=}\PY{l+m+mi}{15}\PY{p}{)}
\PY{n}{title}\PY{p}{(}\PY{l+s+s2}{\PYZdq{}}\PY{l+s+s2}{Error Bar}\PY{l+s+s2}{\PYZdq{}}\PY{p}{)}
\PY{c+c1}{\PYZsh{}adding annotations}
\PY{n}{annotate}\PY{p}{(}\PY{l+s+s2}{\PYZdq{}}\PY{l+s+s2}{Noisy Data}\PY{l+s+s2}{\PYZdq{}}\PY{p}{,}\PY{p}{(}\PY{n}{t}\PY{p}{[}\PY{l+m+mi}{0}\PY{p}{]}\PY{p}{,} \PY{n}{data}\PY{p}{[}\PY{l+m+mi}{0}\PY{p}{,} \PY{l+m+mi}{0}\PY{p}{]}\PY{p}{)}\PY{p}{,} \PY{n}{xytext}\PY{o}{=}\PY{p}{(}\PY{l+m+mi}{40}\PY{p}{,} \PY{o}{\PYZhy{}}\PY{l+m+mi}{40}\PY{p}{)}\PY{p}{,} \PY{n}{textcoords}\PY{o}{=}\PY{l+s+s2}{\PYZdq{}}\PY{l+s+s2}{offset points}\PY{l+s+s2}{\PYZdq{}}\PY{p}{,} \PY{n}{arrowprops}\PY{o}{=}\PY{p}{\PYZob{}}\PY{l+s+s2}{\PYZdq{}}\PY{l+s+s2}{arrowstyle}\PY{l+s+s2}{\PYZdq{}}\PY{p}{:} \PY{l+s+s2}{\PYZdq{}}\PY{l+s+s2}{\PYZhy{}\PYZgt{}}\PY{l+s+s2}{\PYZdq{}}\PY{p}{\PYZcb{}}\PY{p}{)}
\PY{n}{annotate}\PY{p}{(}\PY{l+s+s2}{\PYZdq{}}\PY{l+s+s2}{True Data}\PY{l+s+s2}{\PYZdq{}}\PY{p}{,}\PY{p}{(}\PY{n}{t}\PY{p}{[}\PY{l+m+mi}{0}\PY{p}{]}\PY{p}{,} \PY{n}{g}\PY{p}{(}\PY{n}{t}\PY{p}{[}\PY{l+m+mi}{0}\PY{p}{]}\PY{p}{)}\PY{p}{)}\PY{p}{,} \PY{n}{xytext}\PY{o}{=}\PY{p}{(}\PY{o}{\PYZhy{}}\PY{l+m+mi}{20}\PY{p}{,} \PY{l+m+mi}{35}\PY{p}{)}\PY{p}{,} \PY{n}{textcoords}\PY{o}{=}\PY{l+s+s2}{\PYZdq{}}\PY{l+s+s2}{offset points}\PY{l+s+s2}{\PYZdq{}}\PY{p}{,} \PY{n}{arrowprops}\PY{o}{=}\PY{p}{\PYZob{}}\PY{l+s+s2}{\PYZdq{}}\PY{l+s+s2}{arrowstyle}\PY{l+s+s2}{\PYZdq{}}\PY{p}{:} \PY{l+s+s2}{\PYZdq{}}\PY{l+s+s2}{\PYZhy{}\PYZgt{}}\PY{l+s+s2}{\PYZdq{}}\PY{p}{\PYZcb{}}\PY{p}{)}
\PY{n}{annotate}\PY{p}{(}\PY{l+s+s2}{\PYZdq{}}\PY{l+s+s2}{Noise in data(Red line)}\PY{l+s+s2}{\PYZdq{}}\PY{p}{,}\PY{p}{(}\PY{n}{t}\PY{p}{[}\PY{l+m+mi}{0}\PY{p}{]}\PY{p}{,} \PY{n}{data}\PY{p}{[}\PY{l+m+mi}{0}\PY{p}{,} \PY{l+m+mi}{0}\PY{p}{]}\PY{o}{\PYZhy{}}\PY{l+m+mf}{0.06}\PY{p}{)}\PY{p}{,} \PY{n}{xytext}\PY{o}{=}\PY{p}{(}\PY{o}{\PYZhy{}}\PY{l+m+mi}{20}\PY{p}{,} \PY{o}{\PYZhy{}}\PY{l+m+mi}{60}\PY{p}{)}\PY{p}{,} \PY{n}{textcoords}\PY{o}{=}\PY{l+s+s2}{\PYZdq{}}\PY{l+s+s2}{offset points}\PY{l+s+s2}{\PYZdq{}}\PY{p}{,} \PY{n}{arrowprops}\PY{o}{=}\PY{p}{\PYZob{}}\PY{l+s+s2}{\PYZdq{}}\PY{l+s+s2}{arrowstyle}\PY{l+s+s2}{\PYZdq{}}\PY{p}{:} \PY{l+s+s2}{\PYZdq{}}\PY{l+s+s2}{\PYZhy{}\PYZgt{}}\PY{l+s+s2}{\PYZdq{}}\PY{p}{\PYZcb{}}\PY{p}{)}
\PY{n}{legend}\PY{p}{(}\PY{p}{)}
\PY{n}{grid}\PY{p}{(}\PY{p}{)}
\PY{n}{show}\PY{p}{(}\PY{p}{)}
\end{Verbatim}
\end{tcolorbox}

    \begin{center}
    \adjustimage{max size={0.9\linewidth}{0.9\paperheight}}{output_11_0.png}
    \end{center}
    { \hspace*{\fill} \\}
    
    \hypertarget{part-6}{%
\subsection{Part 6}\label{part-6}}

Here, we are creating the M and p matrices. First, we are generating the
function values for the given time and then fill the values in the M
matrix. The values of matrix p can be anything, so we initialise the
value of p matrix randomly. This function returns the M matrix which
will be used later.

    \begin{tcolorbox}[breakable, size=fbox, boxrule=1pt, pad at break*=1mm,colback=cellbackground, colframe=cellborder]
\prompt{In}{incolor}{7}{\boxspacing}
\begin{Verbatim}[commandchars=\\\{\}]
\PY{k}{def} \PY{n+nf}{Q6}\PY{p}{(}\PY{p}{)}\PY{p}{:}
    \PY{k}{global} \PY{n}{yy}\PY{p}{,} \PY{n}{gt}\PY{p}{,} \PY{n}{t}
    \PY{n}{M} \PY{o}{=} \PY{n}{np}\PY{o}{.}\PY{n}{zeros}\PY{p}{(}\PY{p}{(}\PY{l+m+mi}{101}\PY{p}{,}\PY{l+m+mi}{2}\PY{p}{)}\PY{p}{)}
    \PY{n}{p} \PY{o}{=} \PY{n}{np}\PY{o}{.}\PY{n}{zeros}\PY{p}{(}\PY{p}{(}\PY{l+m+mi}{2}\PY{p}{,}\PY{l+m+mi}{1}\PY{p}{)}\PY{p}{)}
    \PY{c+c1}{\PYZsh{}initiating M matrix with corresponding values}
    \PY{k}{for} \PY{n}{i} \PY{o+ow}{in} \PY{n+nb}{range}\PY{p}{(}\PY{l+m+mi}{101}\PY{p}{)}\PY{p}{:}
        \PY{n}{M}\PY{p}{[}\PY{n}{i}\PY{p}{,}\PY{l+m+mi}{0}\PY{p}{]} \PY{o}{=} \PY{n}{sp}\PY{o}{.}\PY{n}{jv}\PY{p}{(}\PY{l+m+mi}{2}\PY{p}{,}\PY{n}{t}\PY{p}{[}\PY{n}{i}\PY{p}{]}\PY{p}{)} 
        \PY{n}{M}\PY{p}{[}\PY{n}{i}\PY{p}{,} \PY{l+m+mi}{1}\PY{p}{]} \PY{o}{=} \PY{n}{t}\PY{p}{[}\PY{n}{i}\PY{p}{]}
    \PY{c+c1}{\PYZsh{}initiating p matrix with corresponding values}
    \PY{n}{p}\PY{p}{[}\PY{l+m+mi}{0}\PY{p}{,}\PY{l+m+mi}{0}\PY{p}{]} \PY{o}{=} \PY{l+m+mf}{1.0}
    \PY{n}{p}\PY{p}{[}\PY{l+m+mi}{1}\PY{p}{,}\PY{l+m+mi}{0}\PY{p}{]} \PY{o}{=} \PY{l+m+mf}{0.5}

    \PY{c+c1}{\PYZsh{}calculating the ans using matrix multiplication of M and p matrix}
    \PY{n}{ans} \PY{o}{=} \PY{n}{np}\PY{o}{.}\PY{n}{matmul}\PY{p}{(}\PY{n}{M}\PY{p}{,}\PY{n}{p}\PY{p}{)}
    \PY{n}{g\PYZus{}ans} \PY{o}{=} \PY{n}{np}\PY{o}{.}\PY{n}{array}\PY{p}{(}\PY{n}{g}\PY{p}{(}\PY{n}{t}\PY{p}{,}\PY{l+m+mf}{1.0}\PY{p}{,}\PY{l+m+mf}{0.5}\PY{p}{)}\PY{p}{)}\PY{p}{[}\PY{p}{:}\PY{p}{,}\PY{k+kc}{None}\PY{p}{]}

    \PY{n}{chk\PYZus{}ans} \PY{o}{=} \PY{n}{np}\PY{o}{.}\PY{n}{array\PYZus{}equal}\PY{p}{(}\PY{n}{ans}\PY{p}{,} \PY{n}{g\PYZus{}ans}\PY{p}{)}
    \PY{c+c1}{\PYZsh{}print(chk\PYZus{}ans)}
    \PY{k}{return} \PY{n}{M}
\end{Verbatim}
\end{tcolorbox}

    \hypertarget{part-7-8}{%
\subsection{Part 7-8}\label{part-7-8}}

In this part, we are trying to find the values of A ans B which has the
least mean square error. We are trying out different values of A and B.
Then we are plotting the contours of error we get for those different
values of A and B. Here, from the plot we can see that the error value
converges to 0 at the point A=1.05, B=-0.105.

    \begin{tcolorbox}[breakable, size=fbox, boxrule=1pt, pad at break*=1mm,colback=cellbackground, colframe=cellborder]
\prompt{In}{incolor}{8}{\boxspacing}
\begin{Verbatim}[commandchars=\\\{\}]
\PY{c+c1}{\PYZsh{}intialising some values of A and B}
\PY{n}{A\PYZus{}chk} \PY{o}{=} \PY{n}{linspace}\PY{p}{(}\PY{l+m+mi}{0}\PY{p}{,}\PY{l+m+mi}{2}\PY{p}{,}\PY{l+m+mi}{21}\PY{p}{)}
\PY{n}{B\PYZus{}chk} \PY{o}{=} \PY{n}{linspace}\PY{p}{(}\PY{o}{\PYZhy{}}\PY{l+m+mf}{0.2}\PY{p}{,}\PY{l+m+mi}{0}\PY{p}{,}\PY{l+m+mi}{21}\PY{p}{)}
\PY{n}{e} \PY{o}{=} \PY{n}{np}\PY{o}{.}\PY{n}{zeros}\PY{p}{(}\PY{p}{(}\PY{n+nb}{len}\PY{p}{(}\PY{n}{A\PYZus{}chk}\PY{p}{)}\PY{p}{,} \PY{n+nb}{len}\PY{p}{(}\PY{n}{B\PYZus{}chk}\PY{p}{)}\PY{p}{)}\PY{p}{)}

\PY{c+c1}{\PYZsh{}calculating error value for each value of A and B}
\PY{k}{for} \PY{n}{i} \PY{o+ow}{in} \PY{n+nb}{range}\PY{p}{(}\PY{n+nb}{len}\PY{p}{(}\PY{n}{A\PYZus{}chk}\PY{p}{)}\PY{p}{)}\PY{p}{:}
    \PY{k}{for} \PY{n}{j} \PY{o+ow}{in} \PY{n+nb}{range}\PY{p}{(}\PY{n+nb}{len}\PY{p}{(}\PY{n}{B\PYZus{}chk}\PY{p}{)}\PY{p}{)}\PY{p}{:}
        \PY{n}{e}\PY{p}{[}\PY{n}{i}\PY{p}{,} \PY{n}{j}\PY{p}{]} \PY{o}{=} \PY{n}{mean}\PY{p}{(}\PY{n}{square}\PY{p}{(}\PY{n}{g}\PY{p}{(}\PY{n}{t}\PY{p}{,}\PY{n}{A\PYZus{}chk}\PY{p}{[}\PY{n}{i}\PY{p}{]}\PY{p}{,}\PY{n}{B\PYZus{}chk}\PY{p}{[}\PY{n}{j}\PY{p}{]}\PY{p}{)}\PY{o}{\PYZhy{}}\PY{n}{yy}\PY{p}{[}\PY{p}{:}\PY{p}{,}\PY{l+m+mi}{0}\PY{p}{]}\PY{p}{)}\PY{p}{)}

\PY{n}{x}\PY{p}{,} \PY{n}{y} \PY{o}{=} \PY{n}{np}\PY{o}{.}\PY{n}{meshgrid}\PY{p}{(}\PY{n}{A\PYZus{}chk}\PY{p}{,} \PY{n}{B\PYZus{}chk}\PY{p}{)}
\PY{c+c1}{\PYZsh{}plotting contours of error}
\PY{n}{co} \PY{o}{=} \PY{n}{contour}\PY{p}{(}\PY{n}{x}\PY{p}{,} \PY{n}{y}\PY{p}{,} \PY{n}{e}\PY{p}{)}
\PY{n}{plot}\PY{p}{(}\PY{p}{[}\PY{l+m+mf}{1.05}\PY{p}{]}\PY{p}{,}\PY{p}{[}\PY{o}{\PYZhy{}}\PY{l+m+mf}{0.105}\PY{p}{]}\PY{p}{,}\PY{l+s+s1}{\PYZsq{}}\PY{l+s+s1}{ro}\PY{l+s+s1}{\PYZsq{}}\PY{p}{)}
\PY{n}{annotate}\PY{p}{(}\PY{l+s+s2}{\PYZdq{}}\PY{l+s+s2}{True Value}\PY{l+s+se}{\PYZbs{}n}\PY{l+s+s2}{ of Minima}\PY{l+s+s2}{\PYZdq{}}\PY{p}{,} \PY{p}{(}\PY{l+m+mf}{1.05}\PY{p}{,} \PY{o}{\PYZhy{}}\PY{l+m+mf}{0.105}\PY{p}{)}\PY{p}{,}\PY{n}{xytext}\PY{o}{=}\PY{p}{(}\PY{o}{\PYZhy{}}\PY{l+m+mi}{50}\PY{p}{,} \PY{o}{\PYZhy{}}\PY{l+m+mi}{30}\PY{p}{)}\PY{p}{,} \PY{n}{textcoords}\PY{o}{=}\PY{l+s+s2}{\PYZdq{}}\PY{l+s+s2}{offset points}\PY{l+s+s2}{\PYZdq{}}\PY{p}{)}
\PY{n}{xlabel}\PY{p}{(}\PY{l+s+s1}{\PYZsq{}}\PY{l+s+s1}{A}\PY{l+s+s1}{\PYZsq{}}\PY{p}{)}
\PY{n}{ylabel}\PY{p}{(}\PY{l+s+s1}{\PYZsq{}}\PY{l+s+s1}{B}\PY{l+s+s1}{\PYZsq{}}\PY{p}{)}
\PY{n}{clabel}\PY{p}{(}\PY{n}{co}\PY{p}{,} \PY{n}{co}\PY{o}{.}\PY{n}{levels}\PY{p}{[}\PY{p}{:}\PY{l+m+mi}{5}\PY{p}{]}\PY{p}{,} \PY{n}{inline}\PY{o}{=}\PY{l+m+mi}{1}\PY{p}{,} \PY{n}{fontsize}\PY{o}{=}\PY{l+m+mi}{8}\PY{p}{)}
\PY{n}{title}\PY{p}{(}\PY{l+s+s1}{\PYZsq{}}\PY{l+s+s1}{Contours of e}\PY{l+s+s1}{\PYZsq{}}\PY{p}{)}
\PY{n}{grid}\PY{p}{(}\PY{p}{)}
\PY{n}{show}\PY{p}{(}\PY{p}{)}
\end{Verbatim}
\end{tcolorbox}

    \begin{center}
    \adjustimage{max size={0.9\linewidth}{0.9\paperheight}}{output_15_0.png}
    \end{center}
    { \hspace*{\fill} \\}
    
    \hypertarget{part-9-11}{%
\subsection{Part 9-11}\label{part-9-11}}

Here, we use data added with noise having different values of sigma to
predict the values of A and B. We are using the scipy function
\emph{lstsq}. This function finds the value of A and B for which the
mean-square is minimum. Then, we are plotting the error in the
calculated the values of A and B with respect to the sigma values. We
would be plotting the error in the calculated values in both normal
scale and log-log scale.

    \begin{tcolorbox}[breakable, size=fbox, boxrule=1pt, pad at break*=1mm,colback=cellbackground, colframe=cellborder]
\prompt{In}{incolor}{9}{\boxspacing}
\begin{Verbatim}[commandchars=\\\{\}]
\PY{n}{M} \PY{o}{=} \PY{n}{Q6}\PY{p}{(}\PY{p}{)}
\PY{n}{AB} \PY{o}{=} \PY{p}{[}\PY{p}{]}
\PY{n}{err\PYZus{}A} \PY{o}{=} \PY{p}{[}\PY{p}{]}
\PY{n}{err\PYZus{}B} \PY{o}{=} \PY{p}{[}\PY{p}{]}
\PY{c+c1}{\PYZsh{}solving for A and B using different noisy data}
\PY{c+c1}{\PYZsh{}solving is done in a way such that MSE is minimum}
\PY{k}{for} \PY{n}{i} \PY{o+ow}{in} \PY{n+nb}{range}\PY{p}{(}\PY{l+m+mi}{9}\PY{p}{)}\PY{p}{:}
    \PY{n}{tmp} \PY{o}{=} \PY{n}{linalg}\PY{o}{.}\PY{n}{lstsq}\PY{p}{(}\PY{n}{M}\PY{p}{,} \PY{n}{yy}\PY{p}{[}\PY{p}{:}\PY{p}{,}\PY{n}{i}\PY{p}{]}\PY{p}{)}
    \PY{n}{AB}\PY{o}{.}\PY{n}{append}\PY{p}{(}\PY{n}{tmp}\PY{p}{[}\PY{l+m+mi}{0}\PY{p}{]}\PY{p}{)}
    \PY{n}{err\PYZus{}A}\PY{o}{.}\PY{n}{append}\PY{p}{(}\PY{n}{square}\PY{p}{(}\PY{n}{tmp}\PY{p}{[}\PY{l+m+mi}{0}\PY{p}{]}\PY{p}{[}\PY{l+m+mi}{0}\PY{p}{]}\PY{o}{\PYZhy{}}\PY{l+m+mf}{1.05}\PY{p}{)}\PY{p}{)}
    \PY{n}{err\PYZus{}B}\PY{o}{.}\PY{n}{append}\PY{p}{(}\PY{n}{square}\PY{p}{(}\PY{n}{tmp}\PY{p}{[}\PY{l+m+mi}{0}\PY{p}{]}\PY{p}{[}\PY{l+m+mi}{1}\PY{p}{]}\PY{o}{+}\PY{l+m+mf}{0.105}\PY{p}{)}\PY{p}{)}
\PY{c+c1}{\PYZsh{}plotting error in A and in normal scale}
\PY{n}{plot}\PY{p}{(}\PY{n}{scl}\PY{p}{,}\PY{n}{err\PYZus{}A}\PY{p}{,}\PY{l+s+s1}{\PYZsq{}}\PY{l+s+s1}{\PYZhy{}\PYZhy{}ro}\PY{l+s+s1}{\PYZsq{}}\PY{p}{,}\PY{n}{label}\PY{o}{=}\PY{l+s+s1}{\PYZsq{}}\PY{l+s+s1}{Error in A}\PY{l+s+s1}{\PYZsq{}}\PY{p}{)}
\PY{n}{plot}\PY{p}{(}\PY{n}{scl}\PY{p}{,}\PY{n}{err\PYZus{}B}\PY{p}{,}\PY{l+s+s1}{\PYZsq{}}\PY{l+s+s1}{\PYZhy{}\PYZhy{}go}\PY{l+s+s1}{\PYZsq{}}\PY{p}{,}\PY{n}{label}\PY{o}{=}\PY{l+s+s1}{\PYZsq{}}\PY{l+s+s1}{Error in B}\PY{l+s+s1}{\PYZsq{}}\PY{p}{)}
\PY{n}{xlabel}\PY{p}{(}\PY{l+s+s1}{\PYZsq{}}\PY{l+s+se}{\PYZbs{}u03C3}\PY{l+s+s1}{\PYZsq{}}\PY{p}{)}
\PY{n}{ylabel}\PY{p}{(}\PY{l+s+s1}{\PYZsq{}}\PY{l+s+s1}{Error}\PY{l+s+s1}{\PYZsq{}}\PY{p}{)}
\PY{n}{title}\PY{p}{(}\PY{l+s+s1}{\PYZsq{}}\PY{l+s+s1}{Error in normal scale}\PY{l+s+s1}{\PYZsq{}}\PY{p}{)}
\PY{n}{grid}\PY{p}{(}\PY{p}{)}
\PY{n}{legend}\PY{p}{(}\PY{p}{)}
\PY{n}{show}\PY{p}{(}\PY{p}{)}

\PY{c+c1}{\PYZsh{}plotting error in A and in log\PYZhy{}log scale}
\PY{n}{x} \PY{o}{=} \PY{n}{logspace}\PY{p}{(}\PY{o}{\PYZhy{}}\PY{l+m+mi}{1}\PY{p}{,}\PY{o}{\PYZhy{}}\PY{l+m+mi}{3}\PY{p}{,}\PY{l+m+mi}{9}\PY{p}{)}
\PY{n}{loglog}\PY{p}{(}\PY{n}{x}\PY{p}{,}\PY{n}{err\PYZus{}A}\PY{p}{,}\PY{l+s+s1}{\PYZsq{}}\PY{l+s+s1}{ro}\PY{l+s+s1}{\PYZsq{}}\PY{p}{)}
\PY{n}{loglog}\PY{p}{(}\PY{n}{x}\PY{p}{,}\PY{n}{err\PYZus{}B}\PY{p}{,}\PY{l+s+s1}{\PYZsq{}}\PY{l+s+s1}{go}\PY{l+s+s1}{\PYZsq{}}\PY{p}{)}
\PY{n}{errorbar}\PY{p}{(}\PY{n}{x}\PY{p}{,}\PY{n}{err\PYZus{}A}\PY{p}{,}\PY{l+m+mf}{0.0005}\PY{p}{,}\PY{n}{fmt}\PY{o}{=}\PY{l+s+s1}{\PYZsq{}}\PY{l+s+s1}{ro}\PY{l+s+s1}{\PYZsq{}}\PY{p}{,}\PY{n}{label}\PY{o}{=}\PY{l+s+s1}{\PYZsq{}}\PY{l+s+s1}{errbar\PYZus{}A}\PY{l+s+s1}{\PYZsq{}}\PY{p}{)}
\PY{n}{errorbar}\PY{p}{(}\PY{n}{x}\PY{p}{,}\PY{n}{err\PYZus{}B}\PY{p}{,}\PY{l+m+mf}{0.0005}\PY{p}{,}\PY{n}{fmt}\PY{o}{=}\PY{l+s+s1}{\PYZsq{}}\PY{l+s+s1}{go}\PY{l+s+s1}{\PYZsq{}}\PY{p}{,}\PY{n}{label}\PY{o}{=}\PY{l+s+s1}{\PYZsq{}}\PY{l+s+s1}{errbar\PYZus{}B}\PY{l+s+s1}{\PYZsq{}}\PY{p}{)}
\PY{n}{xlabel}\PY{p}{(}\PY{l+s+s1}{\PYZsq{}}\PY{l+s+se}{\PYZbs{}u03C3}\PY{l+s+s1}{\PYZsq{}}\PY{p}{)}
\PY{n}{ylabel}\PY{p}{(}\PY{l+s+s1}{\PYZsq{}}\PY{l+s+s1}{Error}\PY{l+s+s1}{\PYZsq{}}\PY{p}{)}
\PY{n}{title}\PY{p}{(}\PY{l+s+s1}{\PYZsq{}}\PY{l+s+s1}{Error in log\PYZhy{}log scale}\PY{l+s+s1}{\PYZsq{}}\PY{p}{)}
\PY{n}{legend}\PY{p}{(}\PY{p}{)}
\PY{n}{grid}\PY{p}{(}\PY{p}{)}
\PY{n}{show}\PY{p}{(}\PY{p}{)}
\end{Verbatim}
\end{tcolorbox}

    \begin{center}
    \adjustimage{max size={0.9\linewidth}{0.9\paperheight}}{output_17_0.png}
    \end{center}
    { \hspace*{\fill} \\}
    
    \begin{center}
    \adjustimage{max size={0.9\linewidth}{0.9\paperheight}}{output_17_1.png}
    \end{center}
    { \hspace*{\fill} \\}
    
    \hypertarget{conclusion}{%
\section{Conclusion}\label{conclusion}}

From the above experiment, we found the error generated for different
values of the standard deviation of the noise. We can see from the plots
that the value of log of error varies linearily with the log of standard
deviation of the noise.


    % Add a bibliography block to the postdoc
    
    
    
\end{document}
